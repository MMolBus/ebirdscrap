% Options for packages loaded elsewhere
\PassOptionsToPackage{unicode}{hyperref}
\PassOptionsToPackage{hyphens}{url}
%
\documentclass[
]{article}
\usepackage{lmodern}
\usepackage{amssymb,amsmath}
\usepackage{ifxetex,ifluatex}
\ifnum 0\ifxetex 1\fi\ifluatex 1\fi=0 % if pdftex
  \usepackage[T1]{fontenc}
  \usepackage[utf8]{inputenc}
  \usepackage{textcomp} % provide euro and other symbols
\else % if luatex or xetex
  \usepackage{unicode-math}
  \defaultfontfeatures{Scale=MatchLowercase}
  \defaultfontfeatures[\rmfamily]{Ligatures=TeX,Scale=1}
\fi
% Use upquote if available, for straight quotes in verbatim environments
\IfFileExists{upquote.sty}{\usepackage{upquote}}{}
\IfFileExists{microtype.sty}{% use microtype if available
  \usepackage[]{microtype}
  \UseMicrotypeSet[protrusion]{basicmath} % disable protrusion for tt fonts
}{}
\makeatletter
\@ifundefined{KOMAClassName}{% if non-KOMA class
  \IfFileExists{parskip.sty}{%
    \usepackage{parskip}
  }{% else
    \setlength{\parindent}{0pt}
    \setlength{\parskip}{6pt plus 2pt minus 1pt}}
}{% if KOMA class
  \KOMAoptions{parskip=half}}
\makeatother
\usepackage{xcolor}
\IfFileExists{xurl.sty}{\usepackage{xurl}}{} % add URL line breaks if available
\IfFileExists{bookmark.sty}{\usepackage{bookmark}}{\usepackage{hyperref}}
\hypersetup{
  pdftitle={ebird.crap.chl},
  pdfauthor={Manuel Molina-Bustamente},
  hidelinks,
  pdfcreator={LaTeX via pandoc}}
\urlstyle{same} % disable monospaced font for URLs
\usepackage[margin=1in]{geometry}
\usepackage{graphicx}
\makeatletter
\def\maxwidth{\ifdim\Gin@nat@width>\linewidth\linewidth\else\Gin@nat@width\fi}
\def\maxheight{\ifdim\Gin@nat@height>\textheight\textheight\else\Gin@nat@height\fi}
\makeatother
% Scale images if necessary, so that they will not overflow the page
% margins by default, and it is still possible to overwrite the defaults
% using explicit options in \includegraphics[width, height, ...]{}
\setkeys{Gin}{width=\maxwidth,height=\maxheight,keepaspectratio}
% Set default figure placement to htbp
\makeatletter
\def\fps@figure{htbp}
\makeatother
\setlength{\emergencystretch}{3em} % prevent overfull lines
\providecommand{\tightlist}{%
  \setlength{\itemsep}{0pt}\setlength{\parskip}{0pt}}
\setcounter{secnumdepth}{-\maxdimen} % remove section numbering
\ifluatex
  \usepackage{selnolig}  % disable illegal ligatures
\fi

\title{ebird.crap.chl}
\author{Manuel Molina-Bustamente}
\date{18/3/2022}

\begin{document}
\maketitle

\hypertarget{ebird.scrap.chl}{%
\subsection{ebird.scrap.chl}\label{ebird.scrap.chl}}

\textbf{Objective}: Obtain complete lists of ofservations for all
species sigthed from one place and dates using web scraping on eBird
webpage.

The function \emph{ebird.scrap.chl} transform species observations from
eBird checklist into individual species observations. The checklists are
obtained from an ebird region (or hotspot) in a set of dates, without
date limits.

\emph{ebird.scrap.chl} uses \textbf{rebird}::\emph{ebirdchecklistfeed}
to obtain the checklist, so \textbf{it is neccesary to have a key from
eBird API}. To obtain an eBird key you need to have an account in the
platform. Once you have an account you can obtain your key in
\url{https://ebird.org/api/keygen} You have more information of about
eBird API 2.0 in
\url{https://documenter.getpostman.com/view/664302/S1ENwy59\#4e020bc2-fc67-4fb6-a926-570cedefcc34}

Is highly recomended to explore rebird package functions. You could
visit
\url{https://cran.r-project.org/web/packages/rebird/vignettes/rebird_vignette.html}

\hypertarget{arguments}{%
\subsubsection{Arguments:}\label{arguments}}

\begin{itemize}
\item
  \textbf{loc}: (required) Region code or locID (if a hotspot). Region
  code can be country code (e.g.~``ES''), subnational1 code
  (states/provinces, e.g.~``ES-CL''), or subnational2 code (counties,
  e.g.~``US-VA-003''). You could find country codes in
  \url{https://www.iso.org/obp/ui/\#home}.
\item
  \textbf{dates}: (required) Date of historic observation date formatted
  according to ISO 8601 (e.g.~'YYYY-MM-DD', or 'YYYY-MM-DD hh:mm').
  Hours and minutes are excluded.
\item
  \textbf{ebird\_key}: eBird API key. You can obtain one from
  \url{https://ebird.org/api/keygen}. We strongly recommend storing it
  in your .Renviron file as an environment variable called EBIRD\_KEY
\end{itemize}

\end{document}
